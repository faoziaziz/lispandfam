% Created 2020-02-14 Fri 14:44
% Intended LaTeX compiler: pdflatex
\documentclass[11pt]{article}
\usepackage[utf8]{inputenc}
\usepackage[T1]{fontenc}
\usepackage{graphicx}
\usepackage{grffile}
\usepackage{longtable}
\usepackage{wrapfig}
\usepackage{rotating}
\usepackage[normalem]{ulem}
\usepackage{amsmath}
\usepackage{textcomp}
\usepackage{amssymb}
\usepackage{capt-of}
\usepackage{hyperref}
\author{John Doe}
\date{\today}
\title{lispandfam}
\hypersetup{
 pdfauthor={John Doe},
 pdftitle={lispandfam},
 pdfkeywords={},
 pdfsubject={},
 pdfcreator={Emacs 26.3 (Org mode 9.4)}, 
 pdflang={English}}
\begin{document}

\maketitle
\tableofcontents


\section{Lisp dan Keluarganya}
\label{sec:org3e24ba5}

\subsection{Pendahuluan}
\label{sec:org77fdb14}
\subsubsection{Sejarah LISP}
\label{sec:orga49d034}
LISP adalah bahasa pemrograman yang diciptakan oleh dan sering digunakan untuk AI.
Termasuk OOP karena mendukung Class dan fungsi kelas seperti Inheritance, Polimorfisme, 
dan sebagainya.
\subsubsection{Mengapa LISP}
\label{sec:orgf43d15c}
Lebih mudah untuk mahami komputer
\subsubsection{Mode Dalam LISP}
\label{sec:org4264ecc}
\begin{enumerate}
\item Interpreter
\label{sec:org050fb3d}
(+ 1 1)
\item Editor
\label{sec:org6c63976}
dengan ekstensi lsp.
\end{enumerate}
\subsection{MIT SCHEME}
\label{sec:org5fdeaf6}
\subsubsection{Matematika}
\label{sec:orgf126414}
\textbf{*}
\end{document}
